\documentclass[10pt]{res}
\usepackage[dvipsnames]{xcolor}
\usepackage{helvet,hyperref,graphicx,wrapfig,lipsum,multirow,multicol}
\usepackage[utf8]{inputenc}
\usepackage[left=0.5cm,top=1cm,right=3cm,bottom=1.5cm]{geometry}
\usepackage[spanish]{babel}

\definecolor{link}{HTML}{04396f}

\begin{document}
\begin{resume}
{\Huge \textbf{Giorgio Reveco Barraza}}\\ 
giorgio.reveco@gmail.com / (+56) 9 9997 1480 / La Serena, Chile.\\
\href{https://www.linkedin.com/in/giorgio-reveco/}{\color{link}{https://www.linkedin.com/in/giorgio-reveco/}} \\ 
Última Actualización \today

\textbf{PERFIL PROFESIONAL}\\
\rule[1mm]{\textwidth}{0.5mm}\\ 
\textit{Soy Licenciado en Física y Magíster en Ingeniería Industrial. Me dedico principalmente a la docencia universitaria, actividad que compagino y complemento con mi proyecto de divulgación de las Ciencias Físicas y Matemáticas: Topos Uranos. Cuento con conocimientos avanzados en matemática y física, habilidad para emprender nuevos aprendizajes sin importar su complejidad y de compartir lo aprendido con otros. Nunca desaprovecho la oportunidad de aprender y aplicar algo nuevo.}

\textbf{EXPERIENCIA}\\
\rule[1mm]{\textwidth}{0.5mm}\\ 
{PROYECTO: TOPOS URANOS \hfill (2020 - Presente)}\\
\rule[1mm]{\textwidth}{0.3mm}\\ 
\textbf{Founder \& CEO:} Topos Uranos es un proyecto de divulgación científica que además espera hacer del conocimiento especializado en matemática y física algo accesible para todo el mundo hispanohablante.\\ 
\textbf{Webmaster:} Administro y doy soporte al sitio web del proyecto (\href{https://toposuranos.com}{https://toposuranos.com}).\\
\textbf{Productor \& Creativo:} Redacto las entradas de la página web y  realizo la producción audiovisual para el canal de youtube.(\href{https://www.youtube.com/c/ToposUranos}{https://www.youtube.com/c/ToposUranos}).

{DOCENTE EN EDUCACIÓN SUPERIOR \hfill (2016 - Presente)}\\
\rule[1mm]{1\textwidth}{0.3mm}
\textbf{Como académico he dictado las cátedras de pregrado para los cursos de las unidades e instituciones académicas que se muestran a continuación:}

{\textbf{(2019 - Presente)} {\sc Instituto Profesional INACAP}}  \hfill La Serena, Chile\\ 
{Mecánica de Fluidos, Termodinámica, Análisis Estadístico de Datos, Tecnología Aplicada a la Logística, Física Mecánica, Óptica, Electromagnetismo, Matemática y Física General} para las áreas de Construcción y de Procesos Industriales.

{\textbf{(2016 - Presente)} {\sc Universidad de La Serena} } \hfill La Serena, Chile\\
{Matemática, Álgebra General, Introducción al Cálculo, Cálculo I, II y III, Probabilidades y Estadística, Introducción a los Métodos Matemáticos para la Física y Física Moderna (ejercicios). También he dirigido las actividades de Laboratorio para el curso de Ondas.} para el Departamento de Matemáticas y el Departamento de Física y Astronomía.

{\textbf{(2018 - 2020)} {\sc Universidad Central de Chile} } \hfill Santiago y La Serena, Chile\\
{Física General (para carreras Advance), Nivelación Matemática y Cálculo. También he dirigido las actividades de Laboratorio para el curso de Introducción a la Física} para el Departamento de Ingeniería y Arquitectura(Sede La Serena) y el Departamento de Ciencias de la Salud (Sede Santiago).

{\textbf{(2020)} {\sc Universidad del Alba} (Ex Pedro de Valdivia)} \hfill La Serena, Chile\\
Bioestadística para la Facultad de Ciencias de la Salud.

{DOCENTE EN EDUCACIÓN MEDIA/SECUNDARIA \hfill (2017 - 2018)}\\
\rule[1mm]{1\textwidth}{0.3mm}\\
\textbf{Durante este período trabajé además como profesor de educación media y preparador PSU para cursos de matemática y física  en las instituciones que se nombran a continuación.}

{\textbf{(2018)} {\sc Centro Educativo Max Planck \& CEALS} } \hfill  La Serena, Chile\\
{\textbf{(2017)} {\sc Preuniversitario PreuGauss} } \hfill La Serena, Chile\\

{AYUDANTE DE CÁTEDRA \hfill (2012 - 2015)}\\
\rule[1mm]{1\textwidth}{0.3mm}\\
\textbf{Durante este período compaginé mis estudios de Licenciatura en Física con actividades de ayudantías, en donde colaboré con otros docentes dictando clases de reforzamiento y asistencia en laboratorios.}

{\textbf{(2012 - 2015)} {\sc Departamento de Física y Astronomía, Universidad de La Serena} \hfill  } \hfill  La Serena, Chile\\
Clases de ayudantia para: Introducción a la Física y Electromagnetismo. Colaboración en laboratorio para los cursos de {Mecánica} y {Electrónica.}



\begin{minipage}[t]{0.3\textwidth}
\textbf{COMPTENCIAS}\\
\rule[1mm]{\textwidth}{0.5mm}\\ 
\vspace{2.3px}
\begin{tabular}{lll}
\multicolumn{3}{l}{IDIOMAS}\\
\textbf{Espa\~nol} &:& Nativo\\
\textbf{Inglés} &:& Medio\\
\textbf{Chino} \textbf{Mandarín} &:& Inicial\\ \\

\multicolumn{2}{l}{PROGRAMACIÓN}\\
\textbf{C++} &:& Medio\\
\textbf{Python} &:& Medio\\
\textbf{Wolfram} &:& Medio\\ 
\textbf{R} &:& Básico\\
\textbf{JavaScript} &:& Medio\\ \\


\multicolumn{3}{l}{INFORMÁTICA}\\ 
\textbf{\LaTeX} &:& Avanzado\\
\textbf{Excel} &:& Medio\\
\textbf{Wordpress} &:& Medio\\
\textbf{Moodle} &:& Medio \\
\textbf{SMF} &:& Medio
\end{tabular} 

\end{minipage} \hspace{0.02\textwidth} \begin{minipage}[t]{0.66\textwidth}
\textbf{EDUCACIÓN Y FORMACIÓN}\\
\rule[1mm]{\textwidth}{0.5mm}\\ 
%{\textbf{Escuela de Negocios Europea de Barcelona}} \hfill Barcelona, España \\ {MBA – Máster(c) en Administración y Dirección de Empresas}\hfill (2021-2023)\\ {Máster(c) en Big Data y Business Intelligence}\hfill (2021-2023)\\ {Máster(c) en Project Management} \hfill (2021-2023)\\ {Diploma(c) de Especialización en Coaching y PNL} \hfill (2021-2023)\\  { Business English Program Certificate(c)}\hfill (2021-2023) \\

\textbf{Universidad Central de Chile} \hfill La Serena, Chile \\
{Magister en Ingeniería Industrial} \hfill {2021}\\
{Diplomado en Gestión de la Innovación} \hfill {2020}\\
{Diplomado en Gestión de Operaciones y Log\'istica} \hfill {2019}\\
{Diplomado en Gestión de Proyectos} \hfill {2019}\\

\textbf{Universidad de La Serena} \hfill La Serena, Chile \\
{Diplomado en Docencia en Educación Superior} \hfill {2018}\\
{Licenciado en Física con Mención en Física Aplicada} \hfill {2015}\\

%\textbf{Colegio Libertador Simón Bolivar} \hfill La Serena, Chile \\ {Educación Media Científico Humanista} \hfill {2005}
\end{minipage}\\

\textbf{OTRAS CERTIFICACIONES}\\
\rule[1mm]{\textwidth}{0.5mm}\\ 
%(2020){\textbf{ \href{https://credentials.edx.org/records/programs/shared/77b44c7fdeaa45bda4a66c40bf2fcaa2/}{\color{link}Professional Certificate(c) - Data Science - 78\%}}}\hfill  {\sc HarvardX}\\
{(2019)} \href{https://courses.edx.org/certificates/748eafe6ae214f558d2ade26a9b3b260}{\textbf{\color{link}{Mandarin Chinese Essentials}}}\hfill  {\sc MandarinX}\\
{(2018)} \href{https://courses.edx.org/certificates/328420505fd04bdc8b96162581af1f65}{\textbf{\color{link}{Building Interactive Prototypes using JavaScript}}}\hfill  {\sc Microsoft}\\
{(2018)} \href{https://courses.edx.org/certificates/7bf76ed78c21492896bf9e6f072a4f3b}{\textbf{\color{link}{Designing a Technical Solution}}}\hfill  {\sc Microsoft}\\
{(2018)} \href{https://credentials.edx.org/credentials/513db2db3f814a4b99079d68bca9cc5e}{\textbf{\color{link}{Professional Certificate: Introduction to Computer Science}}} \hfill {\sc Microsoft}\\
{(2018)} \href{https://courses.edx.org/certificates/4be850b973144987b5042f156614cda2}{\textbf{\color{link}{Intermediate C++}}}\hfill  {\sc Microsoft}\\
\end{resume} 
\end{document}